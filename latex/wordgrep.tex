% !TEX TS-program = pdflatex
% !TEX encoding = UTF-8 Unicode

% This is a simple template for a LaTeX document using the "article" class.
% See "book", "report", "letter" for other types of document.

\documentclass[11pt]{article} % use larger type; default would be 10pt

\usepackage[utf8]{inputenc} % set input encoding (not needed with XeLaTeX)

%%% Examples of Article customizations
% These packages are optional, depending whether you want the features they provide.
% See the LaTeX Companion or other references for full information.

%%% PAGE DIMENSIONS
\usepackage{geometry} % to change the page dimensions
\geometry{a4paper} % or letterpaper (US) or a5paper or....
% \geometry{margin=2in} % for example, change the margins to 2 inches all round
% \geometry{landscape} % set up the page for landscape
%   read geometry.pdf for detailed page layout information

\usepackage{graphicx} % support the \includegraphics command and options

% \usepackage[parfill]{parskip} % Activate to begin paragraphs with an empty line rather than an indent

%%% PACKAGES
\usepackage{booktabs} % for much better looking tables
\usepackage{array} % for better arrays (eg matrices) in maths
\usepackage{paralist} % very flexible & customisable lists (eg. enumerate/itemize, etc.)
\usepackage{verbatim} % adds environment for commenting out blocks of text & for better verbatim
\usepackage{subfig} % make it possible to include more than one captioned figure/table in a single float
% These packages are all incorporated in the memoir class to one degree or another...

%%% HEADERS & FOOTERS
\usepackage{fancyhdr} % This should be set AFTER setting up the page geometry
\pagestyle{fancy} % options: empty , plain , fancy
\renewcommand{\headrulewidth}{0pt} % customise the layout...
\lhead{}\chead{}\rhead{}
\lfoot{}\cfoot{\thepage}\rfoot{}

%%% SECTION TITLE APPEARANCE
\usepackage{sectsty}
\allsectionsfont{\sffamily\mdseries\upshape} % (See the fntguide.pdf for font help)
% (This matches ConTeXt defaults)

%%% ToC (table of contents) APPEARANCE
\usepackage[nottoc,notlof,notlot]{tocbibind} % Put the bibliography in the ToC
\usepackage[titles,subfigure]{tocloft} % Alter the style of the Table of Contents
\renewcommand{\cftsecfont}{\rmfamily\mdseries\upshape}
\renewcommand{\cftsecpagefont}{\rmfamily\mdseries\upshape} % No bold!

%%% END Article customizations

%%% The "real" document content comes below...

\title{wordgrep utility's specifications}
\author{Benoît Dunand-Laisin}
%\date{} % Activate to display a given date or no date (if empty),
         % otherwise the current date is printed 

\begin{document}
\maketitle

\section{Introduction}

\section{Specifications}
Write a utility with the following syntax:

      wordgrep dictfile [inputfile ...]

wordgrep parses a dictionary file as follows:
\begin{itemize}
\item a line with a single word defines a word to be found in the input stream
\item empty lines and line starting with \# are ignored
\end{itemize}
   wordgrep then filters (to the standard output) its input stream, the
   concatenation of all input files specified on the command line or standard
   input if none is given.  wordgrep outputs any line from the input streams that
   contains one of the words from the dictfile. Matching is performed on a full
   word basis.  Words are delimited by white space.

   Implementation should be simple and efficient, dictionary size should be
   limited only by memory,

      An arbitrary limit on line length is OK for the dictionary, but should
   be unnecessary on input stream. 

\section{Existing alternatives}
Why not try this:
grep -wf dictfile [inputfile ...]

\section{Test cases}

\subsection{Without any parameter}
\begin{tabular}{|c|c|}
\hline
Command line &
wordgrep\\
\hline
dictfile content &
n/a\\
\hline
file(s) content &
n/a\\
\hline
Error output &
Without a dictionnary file, there is nothing to do\\
\hline
Standard output &
No output\\
\hline
\end{tabular}

\subsection{With an empty dictfile and one file}
\begin{tabular}{|c|c|}
\hline
Command line &
wordgrep dictfileEmpty fileA\\
\hline
dictfile content &
\\
\hline
file(s) content &
Never mind\\
\hline
Error output &
Dictionnary file is empty, there is nothing to do\\
\hline
Standard output &
No output\\
\hline
\end{tabular}

\subsection{With a dictfile and one file}
\begin{tabular}{|c|c|}
\hline
Command line &
wordgrep dictfile fileA\\
\hline
dictfile content &
Toto\\
&TITI\\
\hline
file(s) content & I like Toto's joke.\\
&However, some on them are not funny.\\
&Toto + Toto = Toto's head\\
&Funny isn't it?\\
\hline
Error output &
No error\\
\hline
Standard output &Toto + Toto = Toto's head\\
\hline
\end{tabular}
\\Note: First line doesn't appear in output because the word is Toto's and not Toto.

\subsection{With a dictfile and more than one file}
\begin{tabular}{|c|c|}
\hline
Command line &
wordgrep dictfile fileB fileC\\
\hline
dictfile content &
Toto\\
&TITI\\
\hline
file(s) content & \begin{bf}fileB\end{bf}\\
& I like Toto's joke.\\
&However, some on them are not funny.\\
& \begin{bf}fileC\end{bf}\\
&Toto + Toto = Toto's head\\
&Funny isn't it?\\
\hline
Error output &
No error\\
\hline
Standard output &Toto + Toto = Toto's head\\
\hline
\end{tabular}

\subsection{With a dictfile and no file}
\begin{tabular}{|c|c|}
\hline
Command line &
wordgrep dictfile\\
\hline
dictfile content &
Toto\\
&TITI\\
\hline
file(s) content & n/a\\
\hline
Error output &
No error\\
\hline
Standard output &????\\
\hline
\end{tabular}

\subsection{With a missing dictfile and one file}
\begin{tabular}{|c|c|}
\hline
Command line &
wordgrep dictfile2 fileA\\
\hline
dictfile content &n/a\\
\hline
file(s) content & Never mind\\
\hline
Error output &
The given dictfile is missing.\\
\hline
Standard output &No output\\
\hline
\end{tabular}
\subsection{With a dictfile and one missing file}
\begin{tabular}{|c|c|}
\hline
Command line &
wordgrep dictfile fileAA\\
\hline
dictfile content &Never mind\\
\hline
file(s) content & Never mind\\
\hline
Error output &
fileAA is missing\\
\hline
Standard output &No output\\
\hline
\end{tabular}

\subsection{With a dictfile, one missing file and at least one existing file}
\begin{tabular}{|c|c|}
\hline
Command line &
wordgrep dictfile fileAA fileB\\
\hline
dictfile content &
Toto\\
&TITI\\
\hline
file(s) content & \begin{bf}fileAA missing\end{bf}\\
&\begin{bf}fileB\end{bf}\\
&Toto + Toto = Toto's head\\
&Funny isn't it?\\
\hline
Error output &
fileAA is missing\\
\hline
Standard output &Toto + Toto = Toto's head\\
\hline
\end{tabular}
\subsection{Parsing validation}
\begin{tabular}{|c|c|}
\hline
Command line &
wordgrep dictfileA fileD\\
\hline
dictfile content &
Toto\\
&TITI\\
&   some\\
&\\
&World is mine\\
&\# Funny\\
&averybigwordsosolongthatitswillbeignored\\
&don't\\
\hline
file(s) content & I like Toto's joke.\\
&However, some on them are not funny.\\
&Toto + Toto = Toto's head\\
&Funny isn't it? World is mine !!!\\
&I don't think so.\\
\hline
Error output &
The following dictfile entries are ignored:\\
&World is mine\\
&averybigwordsosolongthatitswillbeignored\\
\hline
Standard output &
However, some on them are not funny.\\
&Toto + Toto = Toto's head\\
&I don't think so.\\
\hline
\end{tabular}
\subsection{Big data testing}
To complete...

\section{Implementation}
\subsection{function parseDictionary}
This function is used to parse and load the dictionary content in memory.
\subsubsection{Words storage}
Words are stored in memory, concatened, and separed by a pipe. That way permit to use the strstr function on the words list.\\
Alternative is to manage an array of char*. However the grep operation would be more time consumming (but it permits to use the full memory capacity).\\
Memory is allocated first for a "big" size. and then reallocated each time it's necessary for a defined size (that doesn't equal to the size of the word to store).\\
Alternative is to reallocate the memory for each word, but this lead to poor performances and a worst memory fragmentation.
\subsubsection{Dictionnary file reading and filtering}
Dictionnary file is read line by line. The variable used to store the line is size limited abitrary. This will be a known limitation.\\
Lines are left and right trimed before tested.\\
Empty lines are ignored.\\
Lines beginning with a "\#" are ignored.\\
Lines containing a white space are ignored with a warning message.
\subsubsection{Dictionnary context}
The context of the dictionnary contains:
\begin{itemize}
\item The number of words stored
\item The size of the biggest word
\end{itemize}

\subsection{function openInput}
\subsection{function readInput}
This function extracts data from input flux. The data is read character by character (getc function)
The character is stored in the current word.
When a white space is read then
If the current line doesn't contain a dictionnary word, then call the grepWord function
Clear the current word and begin a new one
When a new line is read:
If the current line contains a dictionnary word, then go back to the beginning of the line (fsetpos function) et write the line (fwrite)
Store the position in the file (fgetpos function).
When input flux is ended then exit the function

\subsection{function grepWord}
This function handle a single full word and check its existence in the dictionnary.
The word is stored between two pipes and then, the function strstr is used.
Returns "True" if word is found in the dictionnary, else "False".

\section{Improvement}

\section{Packaging}
\end{document}
